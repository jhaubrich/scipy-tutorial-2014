\documentclass{article}

\usepackage{hyperref}
\usepackage{graphicx}
\usepackage{color}
\usepackage{fancyvrb}

% Pygments syntax highlighting codes
<< pygments['pastie.tex'] >>

\title{Replicate or Perish!}
\author{Antoine Leeuwenhoek}

\begin{document}

\maketitle

\section{Abstract}

In the year of 1657 I discovered very small living creatures in rain water.

If you are a true scientist, you are not to believe my verbal description of these
findings. Instead you must pursue the replication of my observations.

The key feature of Scientific work is not ``Novelty'' but ``Reproducibility''.

\section{Introduction}

This article illustrates how to create a ``Reproducible Research'' report.

\section{Input Data}

The input data was taken from a Tralitus Saltator (sand hopper) specimen, found
minding its own business in the beach of Saint Jacut de la Mer in France on
June 23 2014 at 8:10am.

This work was done by Reproducible Research Warriors in training at the Summer
School on Biomedical Imaging.

Image of the specimen where acquired using mobile phone cameras and a drop of
water as a single-lens microscope, as shown in Figure~\ref{inputimage}.

\begin{figure}
\centering
\includegraphics[width=5cm]{TralitusSaltrator.jpg}
\caption{Input image of the Talitrus Saltrator Specimen}
\label{inputimage}
\end{figure}

\section{Image Analysis}

Image analysis was performed with the SimpleITK package.

The eye structure of the Trilitus Saltator was segmented using a region growing
method based on color similarity.

\section{Results}

\subsection{Image Segmentation}

The resulting image of the segmented eye is presented in Figure~\ref{segmentedimage}.

\begin{figure}
\centering
\includegraphics[width=5cm]{SegmentedEyeOverlay.png}
\caption{Segmentation of the Eye, overlaid on the input image}
\label{segmentedimage}
\end{figure}

The image was segmented using the helper module ``eyesize'', that internally
used a region growing segmentation method followed by a distance map. The
module is invoked the following way:

<< d['00.py|idio|l']['segmented'] >>

This module computes both the segmented image, and the estimation of the eye
structure radius.

\subsection{Radius Estimation}

The smallest radius of the eye structure was estimated using the maximum value
of the distance map, and the value found was << d['image-analysis.json'].radius >>.

\subsection{Segmentation Overlay}

The resulting segmented image is a binary image, that we can then overlap on
top of the grayscale version of the input image. This is done with the
following commands:

<< d['00.py|idio|l']['overlay'] >>


\section{History}

\url{https://royalsociety.org/about-us/history/}

The origins of the Royal Society lie in an ``Invisible College'' of natural
philosophers who began meeting in the mid-1640s to discuss the new philosophy
of promoting knowledge of the natural world through observation and experiment,
which we now call science.

The Society was to meet weekly to witness experiments and discuss what we would
now call scientific topics. The first Curator of Experiments was Robert Hooke.
It was Moray who first told the King, Charles II, of this venture and secured
his approval and encouragement. At first apparently nameless, the name The
Royal Society first appears in print in 1661, and in the second Royal Charter
of 1663 the Society is referred to as 'The Royal Society of London for
Improving Natural Knowledge'.

\section{Nullius in Verba}

The Royal Society's motto ``Nullius in verba'' roughly translates as ``take
nobody's word for it''. It is an expression of the determination of Fellows to
withstand the domination of authority and to verify all statements by an appeal
to facts determined by experiment.

\end{document}
